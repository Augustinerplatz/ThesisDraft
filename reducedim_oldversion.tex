\documentclass[honours]{UNSWthesis}
\linespread{1}
\usepackage{amsfonts}
\usepackage{amssymb}
\usepackage{amsthm}
\usepackage{latexsym,amsmath}
\usepackage{graphicx}

%% define some macros
\newcommand{\R}{\mathbb{R}}
\newcommand{\C}{\mathbb{C}}
\newcommand{\Z}{\mathbb{Z}}
\newcommand{\Q}{\mathbb{Q}}
\newcommand{\G}{\mathcal{G}}
%\newcommand{\H}{\mathcal{H}}
\newcommand{\g}{\mathfrak{g}}
\newcommand{\1}{\mathbf{e}_{1}}
\newcommand{\2}{\mathbf{e}_{3}}
\newcommand{\3}{\mathbf{e}_{3}}

\DeclareMathOperator{\image}{image}
\DeclareMathOperator{\alg}{Alg}
\DeclareMathOperator{\spn}{span}


%% new environments

\newcounter{Item}[section]
%\newenvironment{proof}{\noindent {\bf Proof.}\ }{\qed}
\newenvironment{Definition}{\medskip
                            \refstepcounter{Item}
                            \noindent
                           {\bf Definition \thesection.\theItem.}\ }
                           {\medskip}
\newenvironment{Notation}{\medskip
                            \refstepcounter{Item}
                            \noindent
                           {\bf Notation \thesection.\theItem.}\ }
                           {\medskip}
\newenvironment{Theorem}{\medskip
                            \refstepcounter{Item}
                            \noindent
                           {\bf Theorem \thesection.\theItem.}\ %
                            \begingroup \sl}
                           {\endgroup\medskip}
\newenvironment{Proposition}{\medskip
                            \refstepcounter{Item}
                            \noindent
                           {\bf Proposition \thesection.\theItem.}\ %
                            \begingroup \sl}
                           {\endgroup\medskip}
\newenvironment{Corollary}{\medskip
                            \refstepcounter{Item}
                            \noindent
                           {\bf Corollary \thesection.\theItem.}\ %
                            \begingroup \sl}
                           {\endgroup\medskip}
\newenvironment{Lemma}{\medskip
                            \refstepcounter{Item}
                            \noindent
                           {\bf Lemma \thesection.\theItem.}\ %
                            \begingroup \sl}
                           {\endgroup\medskip}
\newenvironment{Conjecture}{\medskip
                            \refstepcounter{Item}
                            \noindent
                           {\bf Conjecture \thesection.\theItem.}\ %
                            \begingroup \sl}
                           {\endgroup\medskip}
\newenvironment{Example}{\medskip
                            \refstepcounter{Item}
                            \noindent
                           {\bf Example \thesection.\theItem.}\ }
                           {\qed}
\newenvironment{Remark}{\smallskip
                            \refstepcounter{Item}
                            \noindent
                           {\bf Remark \thesection.\theItem.}\ }
                           {\qed}
\newenvironment{Question}{\smallskip
                            \refstepcounter{Item}
                            \noindent
                           {\bf Question \thesection.\theItem.}\ }
                           {\par}
\newenvironment{theoremlist}{\begin{list}{}
                        {\setlength{\parsep}{0pt}
                        \setlength{\topsep}{\smallskipamount}} }
                        {\end{list}}

\title{Rigidity of Coset-Preserving Maps on Upper-Triangular Nilpotent Lie Groups}

\authornameonly{Richard Tierney}

\author{\Authornameonly\\{\bigskip}Supervisor: Professor Michael Cowling}

\begin{document}
\maketitle
\section{Reducing the Dimension of an Algebra}
For the rest of the thesis, $\g$ will denote a nilpotent Lie algebra of dimension $n$ with corresponding Lie group $\G$. Suppose the lower central series for $\g$ is $A_{i}$ and the last non-zero term is $A_{N}$. Let $\phi: \g \longrightarrow \g$ denote a linear map that preserves structures corresponding to cosets in the group. \newline
Let $x$ and $y$ be elements of $\g$. If $\alg(x,y)\subsetneq \g$ then by the inductive assumption $\phi|_{\alg(x,y)}$ is either an isomorphism or anti-isomorphism, so $\phi[x,y]= \pm [\phi(x),\phi(y)]$. \newline
Whereas if $\alg(x,y)=\g$, the inductive assumption cannot be directly applied. Therefore we attempt to reduce the dimension of the algebra $\alg(x,y)$ and look at the restrictions obtained by using the inductive assumption on the new algebra. 
\newline Suppose that $\g=\alg(x,y)$.

\begin{Proposition}
Suppose that the terms in the BCH Formula for $x$ and $y$ are labelled $C_{i}$. (So $x=C_{1}$, $y=C_{2}$, $[x,y]=C_{3}$ etc.), and that $C_{k}$ is the last non-zero term in the sequence. 
Then $\{C_{1},\ldots,C_{k}\} $ forms a basis for $\alg(x,y)=\g$.

\end{Proposition}

\begin{proof}
\begin{Lemma}\label{linindepuv}
If $u \in A_{i}\backslash A_{i+1} $ and $v \in A_{i+1}$ then $u$ and $v$ are linearly independent.
\end{Lemma}
\begin{proof}
Since $A_{i+1}$ is a subspace of $\g$, 
\[
\spn \{ v\} \subseteq A_{i+1}.
\]
Hence $u \notin \spn \{ v\}$ so $u$ and $v$ are linearly independent.
%Let $\alpha_{1}u+\alpha_{2}v=0$ for some real numbers $\alpha_{1}$ and $\alpha_{2}$. Then choose some element $z_{1} \in \g$ such that $[z_{1},u] \neq 0$. So 
%\[
%\alpha_{1}[z_{1},u]+\alpha_{2}[z_{1},v]=0.
%\]
%Continue to choose elements $z_{2}$, $z_{3}$ ... $z_{N-i}$ so that $[z_{N-i},[z_{N-i-1},[\ldots[z_{1},u]\ldots]$ is a non-zero element of $A_{N}$. Then 
%\[
%\alpha_{1}[z_{N-i},[z_{N-i-1},[\ldots[z_{1},u]\ldots]+\alpha_{2}[z_{N-i},[z_{N-i-1},[\ldots[z_{1},v]\ldots]=0.
%\]
%But $[z_{N-i},[z_{N-i-1},[\ldots[z_{1},v]\ldots] \in A_{N+1}=0$ so
%\begin{align*}
%\alpha_{1}[z_{N-i},[z_{N-i-1},[\ldots[z_{1},u]\ldots]&=0 \\
%\alpha_{1} &=0.
%\end{align*}
%Hence $\alpha_{2}v=0$ so $\alpha_{2}=0$ and hence $u$ and $v$ are linearly independent. 
\end{proof}

Now suppose that there are real numbers $a_{i}$ such that $a_{1}C_{1}+\ldots + a_{k}C_{k}=0$. Let $a_{r}$ be the first of these coefficients that is non-zero, and suppose that $C_{r} \ldots C_{s}$ all consist of commutators of the same length, i.e. suppose $C_{r} \ldots C_{s} \in A_{i}\backslash A_{i+1}$. Then $C_{s+1} \ldots C_{k} \in A_{i+1}$ so let
\begin{align*}
u &=a_{r}C_{r}+ \ldots a_{s}C_{s} \\
v &=a_{s+1}C_{s+1} + \ldots a_{k}C_{k} \\
\end{align*}
But then $u+v=0$ so $u$ and $v$ are linearly dependent, which contradicts \ref{linindepuv}. Hence there is no non-zero coefficient $a_{j}$. All the coefficients are zero so $\{ C_{1},\ldots,C_{k} \}$ is linearly independent.
\newline
Clearly $\{ C_{1},\ldots,C_{k} \}$ is a spanning set for $\g=\alg(x,y)$ since the $C_{j}$ are all the non-zero commutators involving $x$ and $y$, and by definition $\alg(x,y)$ is anything in the linear span of this set.

\end{proof}

Now that we have a basis for $\alg(x,y)$ it is possible to reduce its dimension by deleting one of the basis elements. We will consider the subalgebras that arise by deleting $x$ ($C_{1}$) or $y$ ($C_{2}$). \newline
Let 
\[
\g_{-x}=\spn \{ C_{2},C_{3},\ldots,C_{k} \}
\]
and
\[
\g_{-y}=\spn \{C_{1},C_{3},C_{4},\ldots,C_{k} \}
\]

\begin{Lemma}\label{gohigher}
If $C_{p}$ and $C_{q}$ are distinct non-zero terms in the sequence $\{C_{i}\}$, then $[C_{p},C_{q}]= C_{p'}-C_{q'}$ where $C_{p'}$ and $C_{q'}$ are also terms in the sequence $\{C_{i}\}$ (they are possibly zero), for which  $p',q' > p$.
\end{Lemma}
\begin{proof}
The proof is by induction, very similar to the proof of \ref{adddim}. Clearly
\begin{align*}
[C_{1},C_{q}]&=[x,C_{q}]\\
&=C_{p'}
\end{align*}
for some $p'>p$. So the result holds if $p=1$. Now suppose the result holds for all distinct $p$ and $q$ with $p \leq K$ ($q$ can be anything distinct from $p$). Then suppose $r \leq K+1$ and that $r \neq q$. Then
\[
C_{r}=[\mu,C_{p}]
\]
for some $\mu \in \{x,y\}$ and some $C_{p}$ with $p < r$ (hence $p\leq K$). 
\newline
So then 
\begin{align*}
[C_{r},C_{q}]&=[[\mu,C_{p}],C_{q}]\\
&=[\mu,[C_{p},C_{q}]]-[C_{p},[\mu,C_{q}]] \\
&=[\mu,C_{p'}]-[C_{p},C_{s}]\\
&=C_{t_{1}}-C_{t_{2}}
\end{align*}
for some $t_{1}>p'>p$ and some $t_{2}>p$ as required. Hence by induction the statement is true for all positive integers $p$ and $q \neq p$.
\end{proof}


\begin{Proposition} The subspaces $\g_{-x}$ and $\g_{-y}$ are Lie subalgebras.
\end{Proposition} 
\begin{proof}
Let $u$ and $v$ be elements of $\g_{-x}$. Then there are real numbers $a_{2}\ldots a_{k}$ and $b_{2} \ldots b_{k}$ such that 
\begin{align*}
u &= a_{2}C_{2}+ a_{3}C_{3} + \ldots a_{k}C_{k} \\
v &= b_{2}C_{2}+ b_{3}C_{3} + \ldots b_{k}C_{k}
\end{align*}
Then $[u,v]\in A_{1}$. Now suppose that 
\[
[u,v]=\gamma_{1}C_{1}+\gamma_{2}C_{2}+\ldots \gamma_{k}C_{k}
\]

\begin{align*}
\gamma_{1}C_{1}+\gamma_{2}C_{2}+\ldots \gamma_{k}C_{k}&=[a_{2}C_{2}+ a_{3}C_{3} + \ldots a_{k}C_{k},b_{2}C_{2}+ b_{3}C_{3} + \ldots b_{k}C_{k}] \\
&=a_{2}b_{3}[C_{2},C_{3}]+a_{2}b_{4}[C_{2},C_{4}]+\ldots a_{k}b_{k-1}[C_{k},C_{k-1}]
\end{align*}
The commutator terms in the last line all begin with a term $C_{i}$ of the sequence $\{C_{i} \}$ with $i \geq 2$. So by \ref{gohigher}, the commutator terms in the last line can be converted into a series of $C_{i}$-terms, all of which have $i \geq 3$. Hence
\[
\gamma_{1}C_{1}+\gamma_{2}C_{2}+\ldots \gamma_{k}C_{k} =\delta_{3}C_{3}+\delta_{4}C_{4}+\ldots \delta_{k}C_{k} 
\]
Since $\{ C_{i} \}$ is linearly independent, the coefficients can be equated, giving $\gamma_{1}=0$. Hence $[u,v] \in \g_{-x}$ and so $\g_{-x}$ is closed under the bracket operation, making it a Lie subalgebra. By symmetry it can be seen that $\g_{-y}$ is also a Lie subalgebra.
\end{proof}

\end{document}